\documentclass[a4paper,oneside,DIV8,10pt]{scrartcl}
  
  \usepackage[a4paper, left=2cm, right=2cm, top=2cm]{geometry}
  
  \usepackage{ngerman}    % fuer die deutschen Trennmuster
  \usepackage[utf8]{inputenc} % falls Sie Umlaute in den Quellen verwenden wollen
  \usepackage[autostyle=true,german=quotes]{csquotes}
  \usepackage{amsmath}   % enthaelt nuetzliche Makros fuer Mathematik
  \usepackage{amsthm}    % fuer Saetze, Definitionen, Beweise, etc.
  \usepackage{relsize}   % fuer \smaller 
  \usepackage{paralist}
  \usepackage{color, colortbl}
  \usepackage{setspace}
  \definecolor{grau}{gray}{0.9}
  \definecolor{ex}{RGB}{0,0,140}
  \renewcommand{\labelitemi}{--}             % aendert die Symbole bei unnumerierten Aufzaehlungen
  \makeatletter                              % Fussnote ohne Symbol
    \def\blfootnote{\xdef\@thefnmark{}\@footnotetext}
  \newcommand{\handouttitle}[4]
   {\begin{center}
      \Large #4 \\
      \small #2
    \end{center}
    \bigskip
    \noindent
    #1 \\ #3
    \hfill
    
    \noindent
    \rule{\linewidth}{.5pt}
    
    \@afterindentfalse\@afterheading
   }
  \makeatother
  \renewcommand{\sectfont}{\normalfont}      

\newcommand{\zit}[1]{%
\fontfamily{phv}\selectfont #1\normalfont}

\begin{document}
\onehalfspacing
  \handouttitle{Universität Trier}
               {}
               {Computerlinguistik}
               {Leitfaden: Die wissenschaftliche Hausarbeit}

\section{Formale Anforderungen und Aufbau einer wissenschaftliche Hausarbeit}

    \subsection{Formale Anforderungen an die Hausarbeit}
        \begin{compactitem}
            \item DIN A4 einseitig
            \item Schriftgröße 12pt, Fußnoten 10pt
            \item keine ausgefallene Schriftart (meist Times New Roman)
            \item Zeilenabstand 1,5-zeilig 
            \item Blocksatz (auf korrekte Silbentrennung achten!)
            \item Absatzkontrolle: Achten Sie darauf, dass die letzte Zeile 
            eines Absatzes nicht erste Zeile einer neuen Seite ist und die 
            erste Zeile eines Absatzes oder eine Kapitelüberschrift nicht 
            am Seitenende steht 
            \item Seitenabstände ca. 3cm
            \item Seitenzahlen unten rechts oder zentriert
            \item Eigenständigkeitserklärung (unterschrieben)
        \end{compactitem}

    \subsection{Formaler Aufbau der Hausarbeit}
        \begin{compactitem}
            \item Titelblatt
            \item Inhaltsverzeichnis
            \item Einleitung
            \item Hauptteil
            \item Schluss
            \item Literaturverzeichnis
            \item Anhang (evtl.)
        \end{compactitem}

    \subsubsection{Titelblatt}
    Wichtig (u.a. für das korrekte Eintragen der Note) ist, dass die 
        Hausarbeit (zusammen mit dem Studienverlaufsbeleg) eindeutig 
        einem Modul bzw. einer Modulprüfung (und natürlich dem/der zu Prüfenden) 
        zugeordnet werden kann. Unwichtig hingegen: Künstlerisch anspruchsvolle 
        Gestaltung, lange, komplizierte Untertitel etc.
        \begin{compactitem}
            \item Universität, Fachbereich und Fach
            \item Seminartitel, Semester, Abgabedatum
            \item Name und Titel des/der Dozenten/Dozentin
            \item Titel der Arbeit (bei Bachelor- und Masterarbeiten muss der 
            Titel exakt dem beim Hochschulprüfungsamt angemeldeten Titel sein)
            \item VerfasserIn: Name, Studienrichtung, Fachsemester, Matrikelnummer, Anschrift, E-Mail
        \end{compactitem}

    \subsubsection{Inhaltsverzeichnis}
        \begin{compactitem}
            \item enthält die Gliederung der Arbeit (Kapitelüberschriften) mit 
            den Seitenzahlen des Beginns der einzelnen Kapitel
            \item Titelblatt und Inhaltsverzeichnis werden bei der 
            Seitennummerierung nicht mitgezählt, Seite 1 ist der Anfang der Einleitung
            \item Seitenzahlen mit arabischen Zahlen 
            \item Seiten mit Verzeichnissen (Abkürzungsverzeichnis etc.) vor der Einleitung 
            (evtl. auch der Anhang) mit römischen Zahlen 
            \item Gliederung mit arabischen oder römischen Zahlen
            \item moderne Textverarbeitungs- oder Textsatzprogramme können 
            bei richtiger Auszeichnung der Gliederung solche Verzeichnisse 
            automatisch erstellen
        \end{compactitem}


    \subsubsection{Einleitung}
        Zentrale Bestandteile der Einleitung sind:
        \begin{compactitem}
            \item Welches Thema bearbeite ich?
            \item Welche inhaltlichen, methodischen und zeitlichen Grenzen
            hat mein Untersuchungsgegenstand und warum?
            \item Welche Leitfrage(n) habe ich für meine Arbeit? Welche 
            Problemstellungen sind damit verbunden? 
            \item Mit welcher Methode erschließe ich das Thema?
            \item Überblick über die Forschungslage zum Thema: Vorstellung der 
            Sekundärliteratur und deren relevante Themen; Gibt es eine Lücke
            in der Forschungslage? Gibt es Forschungskontroversen?
            \item Sinnvolle Einbettung der eigenen Fragestellung in das 
            Forschungsgebiet; Relevanzfrage
            \item Erläuterung und Begründung der Gliederung und des Vorgehens
        \end{compactitem}

    \subsubsection{Hauptteil}
        \begin{compactitem}
            \item Bearbeitung der in der Einleitung formulierten Leitfragen
            anhand von Quellenmaterial
            \item problemorientierte und analytische Auseinandersetzung 
            mit dem Gegenstand
            \item Auseinandersetzung mit der Sekundärliteratur, eigene 
            Positionierung
            \item Bearbeitung der in der Einleitung formulierten Leitfragen
            anhand von Quellenmaterial
            \item problemorientierte und analytische Auseinandersetzung 
            mit dem Gegenstand
            \item Auseinandersetzung mit der Sekundärliteratur, eigene 
            Positionierung
        \end{compactitem}

    \subsubsection{Schluss}
        \begin{compactitem}
            \item auch \enquote{Fazit}, \enquote{Schlussbetrachtung} oder 
            \enquote{Zusammenfassung}
            \item Zusammenführung der in der Einleitung formulierten 
            Leitfragen mit den im Hauptteil entwickelten Ergebnissen der Arbeit
            \item Perspektiven für weitergehende Forschung
            \item eigene Ausführungen an grundsätzliche Fragen anbinden
        \end{compactitem}

    \subsubsection{Literaturverzeichnis}
        \begin{compactitem}
            \item nur die Literatur, die auch wirklich zitiert wird
            \item alphabetisch geordnet (Nachname HerausgeberIn/AutorIn)
            \item korrekt! (Jahr, Seitenzahlen etc.)
        \end{compactitem}

    
    \subsubsection{Anhang}
        \begin{compactitem}
            \item zusätzliche Statistiken / Tabellen
            \item Quellcode oder Teile davon, Hinweis auf veröffentlichte 
            Software, Datensätze, Korpora
        \end{compactitem}
        
\section{Zitierhinweise (APA)}

    \subsection{Zitationen im Text}
        \begin{compactitem}
            \item sinngemäßes, nicht wörtliches Zitat ohne 
            Anführungsstriche, ohne \enquote{vgl.}
            \item Quellenangabe vor dem Satzzeichen
        \end{compactitem}
        \zit{... ist ein Verfahren zur Satzgliedermittlung (Gross, 1990, S. 67)}
    
    \subsubsection{Kurzbelege (Einzelautor)}
        \zit{... beschäftigt sich Saussure (1967) mit der Struktur ...}\\
        \zit{... begründet damit den Strukturalismus (Saussure, 1967).}\\
        \zit{Bereits 1952 nähert sich Glinz dem Problem ...}
        
    \subsubsection{Kurzbelege (mehrere Autoren)}
        \begin{compactitem}
            \item bei zwei Autoren beide zitieren
            \item bei drei bis sechs Autoren beim ersten Mal alle zitieren, 
            danach nur den ersten Autor und \enquote{et al.}
            \item bei mehr als sechs Autoren nur den ersten Autoren zitieren 
        \end{compactitem}
        \zit{(Busch \& Stenschke, 2014)}\\
        \zit{... wie Busch und Stenschke (2014) betonen ...}\\
        \zit{(Carstensen, Ebert, Endriss, Jekat, Klabunde \& Langer, 2001)}\\
        \zit{(Meibauer et al., 2014)}

    \subsubsection{Mehrere Quellen}
        \begin{compactitem}
            \item bei mehrerern Quellen werden die verschiedenen Quellen 
            durch ein Semikolon getrennt und alphabetisch sortiert 
        \end{compactitem}
        \zit{... wurde auf dieser Grundlage die Syntax der 
        Konstituentengrammatik (Fries, 1952; Harris, 1951) entwickelt.}

    \subsubsection{Wörtliche Zitate}
        \begin{compactitem}
            \item wörtliche Zitate sind wortgetreu wiederzugeben und in 
            Anführungszeichen zu setzen
            \item Angabe der Seitenzahl unabdingbar!
        \end{compactitem}
        \zit{... somit können \enquote{bedeutungsverwandte Ausdrücke
        [...] auf ein und denselben Referenten angewandt werden} (Adamzik, 2001, S. 69).}

        \begin{compactitem}
            \item wörtliche Zitate von mehr als 40 Wörtern als eigenen 
            Absatz ohne Anführungszeichen in Blocksatz (eingerückt)
        \end{compactitem}
        \zit{Meibauer et al. (2002) geben folgende Definition:}
        \begin{quote}
        \zit{
        Unter einem Morphem versteht man im Allgemeinen ein einfaches sprachliches
        Zeichen, das nicht mehr in kleinere Einheiten mit bestimmter Lautung 
        und Bedeutung zerlegt werden kann. In diesem Sinne sind Wörter wie 
        \textit{Haus, rot, auf} Morpheme. Morpheme darf man nicht mit 
        Silben verwechseln. Silben haben keine eigene Bedeutung. (S. 29)
        }
        \end{quote}
        
\subsubsection{Sekundärquellen}
        \begin{compactitem}
            \item Sekundärquellen sollte die Ausnahme sein, immer besser
            die Primärquelle konsultieren und zitieren
            \item Quellen kennzeichnen
            \item bei einem Sekundärzitat wird die Primärquelle nicht im
            Literaturverzeichnis aufgelistet
        \end{compactitem}
        \zit{... und dass Wortbildungs-Morphologie nicht regulär 
        sein kann, zeigen bereits Bar-Hillel und Shamir (1960, zitiert
        nach Carstensen et al., 2001, S. 201).}

    \subsection{Literaturverzeichnis}
        \begin{compactitem}
            \item alle (und nur die) verwendeten Materialien anführen 
            \item alphabetisch geordnet, innerhalb chronologisch
            \item nicht nach Quellentyp unterscheiden
            \item Formatierung ohne Aufzählungszeichen mit hängendem Einzug
            \item richtige und vollständige Angaben
            \item Autor(en) und Autor(innen), Erscheinungsjahr, Titel, Erscheinungsangaben
        \end{compactitem}

    \subsubsection{Monographie}
        \begin{compactitem}
            \item Nachname, V. (Jahreszahl). \textit{Titel des Werkes}. Ort: Verlag. 
            \item Zusatzinformationen zum Titel (wie z.B. Auflage oder Band)
            hinter den Titel (nicht kursiv)
        \end{compactitem}
        \hangindent+2em \hangafter=1
        \zit{Gross, H. (1990). \textit{Einführung in die germanistische Linguistik} (2. Aufl.). München: Iudicum-Verlag.}
    
    \subsubsection{Herausgeberwerk}
        \hangindent+2em \hangafter=1
        \zit{Busch, A. \& Wichter, S. (Hrsg.) (2000). \textit{Computerdiskurs und Wortschatz: 
        Corpusanalysen und Auswahlbibliographie}. Frankfurt a.M.}

    \subsubsection{Beitrag in einem Herausgeberwerk / Sammelband}
        \hangindent+2em \hangafter=1
        \zit{Grote, A. \& Schütte, D. (2000). \textit{Entlehnung und 
        Wortbildung im Computerwortschatz - neue Wörter für eine neue Technologie}. 
        In A. Busch \& S. Wichter (Hrsg.), \textit{Computerdiskurs und Wortschatz: 
        Corpusanalysen und Auswahlbibliographie} (S.\,27-121). Frankfurt a.M.}

    \subsubsection{Zeitschriftenartikel}
        \begin{compactitem}
            \item Nachname, V. (Jahr). Titel des Artikels. \textit{Titel der Zeitschrift}, Band(Heft), Seitenzahl.
        \end{compactitem}
        \zit{Knuth, D.\,E. (1968). Semantics of context-free
        languages. \textit{Mathematical Systems Theory, 2}(2), 127-145.}
        
    \subsubsection{Dissertation}
        \hangindent+2em \hangafter=1
        \zit{Koehn, P. (2003). \textit{Noun Phrase Translation}. Ph.D. thesis, 
        University of Southern California.}
        
        
\section{Qualitätskriterien einer wissenschaftlichen Arbeit}

    \subsection{Anforderungen an eine wissenschaftliche Arbeit}
    \begin{compactitem}
        \item behandelt ein eindeutiges Thema, eine klar formulierte 
        Forschungsfrage
        \item trifft über den untersuchten Gegenstand neue Aussagen, 
        betrachtet das Untersuchungsobjekt mindestens aus einem neuen 
        Blickwinkel
        \item ist nützlich, erweitert den Wissens- bzw. Erkenntnisstand 
        in dem zu untersuchenden Forschungsgebiet
        \item ist intersubjektiv nachvollziehbar
        \item zugrunde liegende Quellen müssen nachvollziehbar sein (also 
        vollständig und korrekt aufgeführt)
        \item der zugrunde liegende Forschungs-/Untersuchungsansatz muss
        erkennbar und überprüfbar sein
        \item die Arbeit muss theoriegeleitet sein
        \item strebt allgemeingültige Aussagen an
        \item erfüllt bestimmte Kriterien (Inhalt, Stil und Form betreffend)
    \end{compactitem}
  
    \subsection{Inhalt}
    \begin{compactitem}
        \item Qualität und Relevanz des Themas
        \item Ziel der Arbeit (Beschreibung, Prognose, ...)
        \item Qualität und Quantität der recherchierten Literatur
        \item Nutzung sonstiger Erkenntnisquellen (Daten aus Experimenten etc.)
        \item Aufbau der Arbeit
        \begin{compactitem}
            \item Einleitung: Themenrelevanz, Ziel der Arbeit
            \item Grundlagenteil: Stand des verfügbaren Wissens
            \item Hauptteil: Hypothesenbildung, Konkretisierung von Aussagen, 
            Ableitung von Konsequenzen, Analyse von Daten
            \item Schluss: kritische Würdigung des eigenen Forschungsansatzes
        \end{compactitem}
    \end{compactitem}

    \subsection{Stil}
    \begin{compactitem}
        \item Korrekte Verwendung von Wörtern
        \begin{compactitem}
            \item Verben: Ausdruck, Tempus, Modus, Aktivformulierungen
            \item Substantive: keine Nominalkonstruktionen, keine Pleonasmen
            \item Adjektive: Anzahl und Auswahl (sparsam verwenden)
        \end{compactitem}
        \item wissenschaftliche Diktion (z.B. Verwendung von Fachtermini)
        \item Sprachlogik
        \item Ästhetik der verwendeten Sprache (z.B. Wiederholungen vermeiden)
        \item Prägnanz, Anschaulichkeit, Verständlichkeit (keine langen Schatelsätze)
        \item Lebendigkeit der Präsentation: Wortwahl, Variabilität, Satzbau
    \end{compactitem}

    \subsection{Form}
    \begin{compactitem}
        \item Konsistenz der Gliederung (Kapitel/Unterkapitel)
        \item Zitierweise (Prüfbarkeit der Aussagen)
        \item Rechtschreibung, Grammatik, Zeichensetzung
        \item Quellen im Literaturverzeichnis (vollständig, fehlerfrei, einheitlich, übersichtlich)
        \item Qualität der Präsentation (Abbildungen, Formeln, Symbole)
        \item Schriftsatz
        \item Transparenz und Übersichtlichkeit (Absätze, Hervorhebungen (kursiv, fett), Aufzählungen)
    \end{compactitem}
    
    \subsection{Anforderungen an wissenschaftliche Quellen}
    \begin{compactitem}
        \item die Quellen müssen dem Niveau einer wissenschaftlichen Arbeit entsprechen
        \item Artikel aus allgemeinen Nachschlagewerken oder Zeitungen 
        entsprechen wissenschaftlichen Ansprüchen eher selten
        \item grundsätzlich werden für wissenschaftliche Arbeiten vor allem
        Beiträge aus Fachzeitschriften / Sammelbänden herangezogen 
        (wissenschaftlich, aktuell, qualitativ hochwertig)
        \item zusätzlich können Monographien herangezogen werden 
        (wissenschaftlich, weniger aktuell)
    \end{compactitem}
    
\section{Grundsätze wissenschaftlicher Arbeiten}

    \subsection{Originalität und Eigenständigkeit}
    \begin{compactitem}
        \item steigernde Anforderungen (Seminararbeit - Bachelorarbeit - Masterarbeit - Doktorarbeit)
        \item fremden Gedankengängen und Inhalten vor dem Hintergrund 
        eigener Erkenntnis einen \textit{eigenen} sprachlichen Ausdruck verleihen
        \item fremde Gedanken zu Eigen machen durch Zitate und Verweise
        \item kritische Wertung und Analyse
    \end{compactitem}

    \subsection{Recherche und Zitation}
    \begin{compactitem}
        \item korrekt und sorgfältig recherchieren und zitieren
        \item es muss für den Leser unmißverständlich erkennbar sein, 
        was an geistigem Eigentum übernommen wurde
        \item wörtliche Entlehnungen als Zitat
        \item gedankliche Entlehnungen als Verweis
    \end{compactitem}

    \subsection{Einflüsse}
    \begin{compactitem}
        \item ein objektiver Dritter muss beurteilen können, ob das 
        wissenschaftliche Urteil vollständig unabhängig zustandegekommen ist
        \item alle externen Faktoren offenlegen
        \item Drittmittel, Stipendien , wirtschaftliche Vorteile kenntlich machen
    \end{compactitem}

    \subsection{Zuschreibung von Aussagen}
    \begin{compactitem}
        \item zitierten Autoren keine Aussagen unterstellen, die diese nicht 
        gemacht haben
        \item bei eigenen Übersetzungen fremdsprachlicher Quellen die 
        Originalquelle benennen, auch Übersetzungen Dritter kenntlich machen
        \item insbesondere bei \enquote{sinngemäßer Übersetzung} darauf achten, 
        keine Aussagen zu unterstellen
    \end{compactitem}
 
    \subsection{Fachspezifisches Wissen}
    \begin{compactitem}
        \item tradiertes Allgemeinwissen einer Fachdisziplin muss nicht
        durch Zitierungen nachgewiesen werden
        \item was zu diesem Allgemeinwissen zählt, beurteilt die jeweilige
        Fachdisziplin, im Zweifel die Institution, die die angestrebte 
        Qualifikation bescheinigt (Dozent, Lehrstuhlinhaber, Dekanat)
    \end{compactitem}

    \subsection{Plagiate und Datenmanipulation}
    \begin{compactitem}
        \item die Übernahme fremden geistigen Eigentums ohne entsprechende
        Kenntlichmachung stellt einen Verstoß gegen die Regeln korrekten 
        wissenschaftlichen Arbeitens dar
        \item dies gilt nicht nur für die wörtliche, sondern auch für die 
        gedankliche Übernahme
        \item die Manipulation von Daten ist ebenfalls ein Regelverstoß
        \item Plagiate und Datenmanipulation sind im Regelfall 
        prüfungsrelevante Täuschungsversuche
    \end{compactitem}

    \subsection{Eigene Arbeiten und Texte}
    \begin{compactitem}
        \item die Übernahme eigener Arbeiten und Texte verstößt dann gegen 
        die Regeln wissenschaftlicher Praxis, wenn diese Übernahme in einer
        Qualifikationsarbeit nicht belegt und zitiert wird
        \item Prüfungsordnungen schließen die Wiederverwendung desselben oder 
        ähnlicher Texte desselben Verfassers aus 
    \end{compactitem}

    \subsection{Mehrere Autoren}
    \begin{compactitem}
        \item das Ausgeben von Texten oder Textteilen Dritter - auch wenn 
        der Autor sein Einverständnis gegeben hat (\textit{ghostwriting}) - als eigene Texte ist ein 
        schwerwiegender Verstoß gegen die Regelen wissenschaftlicher Praxis 
        (insbesondere bei Qualifikationsarbeiten)
        \item bei gemeinsamen Arbeiten sind die jeweiligen Autoren kenntlich 
        zu machen
        \item Ehrenautorschaften oder Autorschaften kraft einer hierarchisch
        übergeordneten Position ohne substantiellen Beitrag sind Fehlverhalten
    \end{compactitem}

    \subsection{Verantwortung}
    \begin{compactitem}
        \item die Verantwortung für die Einhaltung der Regeln wissenschaftlichen 
        Arbeitens trägt in erster Linie der Verfasser
        \item die Aufgabe des Betreuers oder Prüfers ist es, die Regeln 
        wissenschaftlichen Arbeitens zu vermitteln und Zweifeln an der 
        Einhaltung nachzugehen
    \end{compactitem}

\end{document}
